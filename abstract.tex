\documentclass[conference]{IEEEtran}
\usepackage{amsthm}
\usepackage{amsfonts}
\usepackage{amsmath}
\usepackage{graphicx}
\usepackage{filecontents}
\usepackage[noadjust]{cite}

\makeatletter
\def\endthebibliography{%
	\def\@noitemerr{\@latex@warning{Empty `thebibliography' environment}}%
	\endlist
}

\newtheorem{proposition}{Proposizione}[section]


\begin{document}
	\DeclareGraphicsExtensions{.png}
	\title{Sulla ricerca del cammino minimo con ostacoli poligonali convessi disgiunti su piano}
	\author{Niccolò Della Rocca}
	\maketitle
	
	\begin{abstract}
		Questo testo descrive ed analizza dati sperimentali relativi ad algoritmi di ricerca applicati su una classe di problemi di rilevanza pratica nella guida autonoma di robots, oltre a descrivere proprietà di questa famiglia di problemi.
	\end{abstract}

	\section{Introduzione}
	Si consideri la seguente classe di problemi: sono assegnati un piano infinito, un punto di partenza S, un punto di 
	destinazione G e si vuole trovare il cammino più breve tra i due. Ogni punto del piano, rappresentato tramite una coppia \textit{(x,y)}, è uno stato e lo spazio degli stati indotto è continuo, infatti stabilita un'origine arbitraria nel piano e due assi cartesiani si può decidere di prendere ad esempio $S=(0,0)$ e $G=(1,0)$ e tra i due esistono infiniti punti, e quindi infiniti stati, anche nel cammino di lunghezza minima, cioè quello che li connette in linea d'aria. Anche i cammini tra $S$ e $G$ sono infiniti, infatti si possono considerare cammini lineari della forma $S\rightarrow M\rightarrow G$ dove $M$ è un qualsiasi punto lungo la bisettrice del segmento $\overline{SG}$, e questi costituiscono una famiglia di cammini parametrizzati dall'angolo con segno $\hat{SMG}$ che è un parametro reale continuo. Reintroducendo la presenza di un insieme di poligoni convessi che fungono da ostacoli il numero di cammini varia a seconda della loro disposizione sulla scena ed al limite potrebbero non esistere cammini tra i due punti selezionati, se tutti quelli possibili in assenza di ostacoli sono adesso occlusi.
	
	\section{Proprietà della soluzione}
	In questa sezione discutiamo informalmente una proprietà del cammino di lunghezza minima nel contesto descritto nella sezione precedente. Tale proprietà è espressa dal seguente risultato: 
	\begin{proposition}\label{prop:cammino_minimo_linea_spezzata}
		Assegnato un insieme di ostacoli poligonali disgiunti su un piano e due punti $S$ e $G$ su di esso, il cammino minimo tra questi consiste in una linea spezzata.
	\end{proposition}
	\begin{figure}[t]
		\centering
		\includegraphics[width=0.5\textwidth]{Images/polygonalPath}
		\label{fig:intuizione_cammino_minimo}
		\caption{Sostituzione sezione curva con segmento, prop. \ref{prop:cammino_minimo_linea_spezzata} }
	\end{figure}
	A sostegno di quanto proposto muoviamo la seguente argomentazione: assumiamo per assurdo che in realtà il cammino $\Gamma$ di lunghezza minima tra $S$ e $G$ sia curvo. In questo caso esiste almeno una circonferenza contenuta nello spazio vuoto che contiene una sezione curva del cammino minimo, e quest'ultima può essere sostituita col segmento che unisce i due punti in cui la curva interseca la circonferenza. Il cammino così ottenuto è localmente più corto rispetto a $\Gamma$ e questo contraddice l'ipotesi che $\Gamma$ sia il cammino minimo, infatti per essere tale deve essere il più breve in ogni sua sottosezione, cioè localmente. Una intuizione del procedimento applicato è illustrata dall'immagine \ref{fig:intuizione_cammino_minimo}.
	
	Una volta convinti della correttezza della proposizione \ref{prop:cammino_minimo_linea_spezzata} è intuibile che vale anche quanto segue: 
	\begin{proposition}\label{prop:cammino_minimo_vertici}
		Assegnato un insieme di ostacoli poligonali disgiunti su un piano e due punti $S$ e $G$ su di esso, il cammino minimo tra questi è una linea spezzata che unisce una sequenza di vertici di poligoni, ad eccezione al più dei punti estremi.
	\end{proposition}
	Non dimostriamo formalmente il risultato; l'intuizione che ci guida in tal senso è che in assenza di ostacoli, oppure se un cammino potesse attraversare questi, quello minimo come già evidenziato sarebbe quello in linea d'aria tra $S$ e $G$. Il fatto che gli ostacoli siano solidi ci impone di aggirarli e la proposizione \ref{prop:cammino_minimo_linea_spezzata} stabilisce che occorre farlo usando linee spezzate. Una qualsiasi linea spezzata che non congiunge due vertici di poligoni può essere sostituita con una più corta avente per estremi due vertici da cui segue \ref{prop:cammino_minimo_vertici}.
	
	\section{Definizione del problema di ricerca}
	I risultati ottenuti fanno uso della formulazione del problema di ricerca descritta in questa sottosezione. Lo spazio degli stati è formato da ogni possibile vettore spostamento da un vertice o punto di partenza o arrivo ad un altro. Lo stato iniziale è il vettore nullo applicato nel punto di partenza ed uno stato è un goal se il suo secondo estremo è il punto di destinazione. Un'azione applicabile in uno stato $\vec{s} = (s_0,s_1)$ è uno qualsiasi dei punti nella scena visibili da $s_1$, ed il risultato di una di queste sarà di cambiare lo stato da $\vec{s}$ a $\vec{s}^{\:'} (s_1, P)$ con $P$ uno dei punti visibili in $s_1$. Il costo di un'azione è definito come la norma euclidea del vettore spostamento risultante dall'applicazione dell'azione stessa, od in altri termini $d_2(s_1, P)$.
	
	\section{Costruzione di una scena}
	In questo testo chiamiamo \emph{scena} una qualsiasi istanza del problema trattato, cioè un piano con un insieme di ostacoli poligonali convessi disgiunti con un punto di partenza ed uno di arrivo, che possono corrispondere a vertici oppure essere punti aggiuntivi. Il programma realizzato per analizzare gli algoritmi è dotato di una funzionalità che consente la generazione di una scena "casuale", ed opera nel modo seguente.
	\begin{enumerate}
		\item Dato un parametro $r_{min}\geq 21$ viene generato un insieme massimale di campioni tramite un processo di \emph{Poisson disk sampling} distanziati l'uno dall'altro almeno di $r_{min}$, che sono salvati in un array.
		\item L'array di campioni viene permutato in maniera casuale e vengono estratti da essi i primi $p$ elementi, con $p$ un parametro impostato esternamente che rappresenta il numero di poligoni da generare.
		\item Per ciascuno punto estratto $p_i$ si calcola la minima distanza $d_i$ dai rimanenti con l'ausilio di un 2D-tree. 
		\item Intorno a ciascun punto si costruisce un cerchio con raggio $21 \leq r < d_i/2$ generato secondo una distribuzione che privilegia raggi più vicini a $d_i/2$. 
		\item Per ciascun cerchio così costruito si generano $v$ angoli casuali la cui somma sia $2\pi$ con $v \leq v_{max}$, parametro impostato esternamente. Questi angoli corrisponderanno ai vertici del poligono generato.
		\item Si sceglie un poligono a caso ed un vertice a caso su di esso come punto di partenza, e si ripete un'altra volta per produrre un punto di destinazione.
	\end{enumerate}
	Il campionamento tramite Poisson disk-sampling garantisce una distribuzione approssimativamente uniforme\cite{BK.KT.PTB.PKV.V16} dei circocentri dei poligoni generati, e costruire circonferenze di raggio massimo $d_i/2$ garantisce che tutti i poligoni generati saranno disgiunti. 
	
	Come suggerito al punto (4) si vogliono privilegiare i raggi più esterni facendo sì che i poligoni generati siano tendenzialmente grandi e costituiscano quindi ostacoli significativi. Per raggiungere questo obiettivo si usa una versione traslata e scalata della distribuzione esponenziale, tale che $\mathbb{P}(r_{min}\leq X < d_i/2) = 1$. Imponendo tale condizione si trova che la densità sarà siffatta:
	\[ \rho_i(r) = 
		\begin{cases}
			0&\text{Se }r < r_{min} \\
			\cfrac{\lambda}{e^{-\lambda r_{min}} - e^{-\lambda d_i/2} }\cdot e^{-\lambda t} &\text{Se }r_{min}\leq r < d_i/2\\
			0&\text{Se }r\geq d_i/2
		\end{cases}
	\]
	Il parametro $\lambda$ è scelto di modo che almeno il $75\%$ della distribuzione cada tra in nella prima metà di $[r_{min}, d_i/2)$. Vista la forma particolare della densità $\rho_i(r)$ si richiede piuttosto che $\mathbb P\left(X \leq 0.5(d_i/2 - r_{min}) \right) \geq 0.75$ dove $\mathbb P_{X}\sim \textsc{Exp}(\lambda)$, il ché impone il seguente lower bound:
	\begin{equation*}
		\lambda \geq \cfrac{1}{0.5\left(d_i/2 - r_{min}\right)}\cdot \ln(\cfrac{1}{1-0.75})
	\end{equation*}
	Un valore $x$ viene campionato dalla distribuzione in questione tramite la tecnica dell'inversione\cite{LD86} ed il raggio che gli corrisponde sarà $d_i/2+r_{min}-x$, in modo da correggere il fatto che la distribuzione prodotta prende valore nella prima metà di $[r_{min}, d_i/2)$ con probabilità di almeno $0.75$ e non nella seconda, come invece si vorrebbe. Vengono inoltre selezionati casualmente due poligoni distinti, e su ciascuno di essi un vertice a caso; i punti così selezionati corrisponderanno ad un punto di partenza e di arrivo casuali. Non è quindi previsto che vengano generati punti estremi negli spazi vuoti da ostacoli in maniera casuale. 
	
	Infine, scegliere i vertici del poligono in maniera casuale lungo le circonferenze generate garantisce che tutti i poligoni prodotti saranno convessi.
	
	\section{Tecnica di sperimentazione}
	
	
	\bibliographystyle{ieeetran}
	\bibliography{references}
	
\end{document}